Esse domínio do MVC é responsável por direcionar os \textit{requests} conforme a regra de negócio, também de acordo com a arquitetura do projeto, ele é responsável por autenticar o usuário e garantir que as rotas do sistema estão sendo acessadas por alguém autenticado. 

No caso dos controladores, também foi criado um específico para cada classe, onde ele será responsável por direcionar os pedidos conforme a especificidade. Dessa forma, podemos utilizar como exemplo o caso da classe \textit{Motoboy}, com o controlador representado no \autoref{alg:controllerMotoboy}.

\begin{lstlisting}[caption={Delivery Routes - Exemplo de um Controller: Motoboy}, style=htmlcssjs, label=alg:controllerMotoboy]
<?php
class MotoboysController extends Controller {
    public function index() {
        $item = Motoboy::all();
        return view('motoboys.index', ['motoboys' => $item]);
    }
    public function create() {
        return view('motoboys.create');
    }
    public function store(MotoboysRequest $request) {
        Motoboy::create($request->all());
        Request()->session()->flash('mensagem-sucesso', 'Registro salvo com sucesso.');
        return redirect()->route('motoboys');
    }
    public function destroy($id) {
        Motoboy::find($id)->delete();
        Request()->session()->flash('mensagem-sucesso', 'Registro excluso com sucesso.');
        return redirect()->route('motoboys');
    }
    public function edit($id) {
        $item = Motoboy::find($id);
        return view('motoboys.edit', compact('item'));
    }
    public function update(MotoboysRequest $request, $id) {
        Motoboy::find($id)->update($request->all());
        Request()->session()->flash('mensagem-sucesso', 'Registro salvo com sucesso.');
        return redirect()->route('motoboys');
    }
}
\end{lstlisting}

\newpage
\begin{table}[H]
    \centering
    \caption{Delivery Routes - Exemplo de router do back-end
    \label{tab:drRoutes}}
\begin{tabular}{llp{9.2cm}l}
\toprule
\textbf{Método} & \textbf{URI} & \textbf{Ação} \\
\midrule
GET & / & Exibe a dashboard do aplicativo. \\
GET & register & Exibe o formulário de inscrição do aplicativo. \\
POST & register & Manipula uma solicitação de registro para o aplicativo. \\
GET & login & Exibe o formulário de login do aplicativo. \\
POST & login & Manipula uma solicitação de login para o aplicativo. \\
POST & password/email & Envia um link de redefinição para o usuário especificado. \\
GET & password/reset & Exibe o formulário para solicitar um link de redefinição de senha. \\
GET & password/reset/\{token\} & Exibe a visualização de redefinição de senha para o token fornecido. \\
POST & password/reset & Redefine a senha do usuário fornecido. \\
POST & logout & Desconecta o usuário do aplicativo. \\
GET & motoboys & Exibe a listagem de motoboys cadastrados. \\
GET & motoboys/create & Exibe o formulário para cadastrar um novo motoboy. \\
POST & motoboys/store & Manipula uma solicitação de cadastro de um motoboy. \\
GET & motoboys/\{id\}/edit & Manipula uma solicitação de edição de um motoboy. \\
PUT & motoboys/\{id\}/update & Manipula uma solicitação de atualização de um motoboy. \\
GET & motoboys/\{id\}/destroy & Manipula uma solicitação de exclusão de um motoboy. \\
GET & payments & Exibe a listagem de pagamentos cadastrados. \\
GET & payments/create & Exibe o formulário para cadastrar um novo pagamento. \\
POST & payments/store & Manipula uma solicitação de cadastro de um pagamento. \\
GET & payments/\{id\}/edit & Manipula uma solicitação de edição de um pagamento. \\
PUT & payments/\{id\}/update & Manipula uma solicitação de atualização de um pagamento. \\
GET & payments/\{id\}/destroy & Manipula uma solicitação de exclusão de um pagamento. \\
GET & orders & Exibe a listagem de pedidos disponíveis para integração. \\
GET & order/\{id\} & Retorna os atributos do pedido desejado em formato JSON. \\
GET & deliveries & Exibe a listagem de entregas cadastradas. \\
GET & deliveries/store & Manipula uma solicitação de cadastro de uma entrega. \\
GET & deliveries/\{id\}/ticket & Manipula uma solicitação de impressão do ticket de uma entrega. \\
GET & delivering/\{id\} & Retorna os atributos dos pedidos da entrega desejada em formato JSON. \\
GET & deliveries/\{id\}/view & Manipula uma solicitação de início de um entrega. \\
GET & deliveries/\{id\}/dispatched & Manipula uma solicitação de despacho de uma entrega. \\
GET & deliveries/\{id\}/concluded & Manipula uma solicitação de conclusão de uma entrega. \\
GET & deliveries/\{id\}/cancelled & Manipula uma solicitação de cancelamento de uma entrega. \\
\bottomrule
\end{tabular}
    \fonte{Autor}
\end{table}

\newpage
Além de implementar os controladores de todas as classes, foi necessário estabelecer quais seriam as rotas e quais controles serão chamados de acordo com a rota. Para isso o \textit{framework} Laravel disponibiliza um \textit{router}, responsável por organizar as rotas e direcioná-las de acordo com o controlador (\autoref{tab:drRoutes}). Nesse mesmo \textit{router}, também direciona as rotas de autenticação caso o usuário não esteja logado.
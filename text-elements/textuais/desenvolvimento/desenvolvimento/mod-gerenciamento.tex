Inicialmente, o objetivo era construir uma aplicação completa para gestão de vendas, focando na geração do pedido e entrega, sendo essa para qualquer ramo de atividade. Porém, descobriu-se a possibilidade de um nicho de mercado ainda não explorado e sem concorrentes, em que não é preciso disputar clientes com grandes empresas renomadas e maduras no assunto.
 
A aplicação desenvolvida foca no gerenciamento das rotas de entregas dos pedidos gerados via e-PDV já operantes, necessitando apenas da disponibilização de uma API no formato de arquivo JSON (\autoref{fig:drHttpAPI}), essa compondo-se de dados essenciais para realização da integração automática abordada na próxima Seção.
 
  \begin{figure}[H]
    \centering
    \caption{Delivery Routes - Requisição HTTP}
    \includegraphics[width=0.6\textwidth]{./dados/figuras/fig16}
    \fonte{\citeonline{iFood}}
    \label{fig:drHttpAPI}
\end{figure}

Primeiramente, é preciso entender os possíveis status e o fluxo de um pedido integrado na Delivery Routes, ambos representados na \autoref{tab:drStatusPedido} e na \autoref{fig:drStatusPedido}.
 
 \begin{table}[H]
    \centering
    \caption{Delivery Routes - Status do pedido
    \label{tab:drStatusPedido}}
\begin{tabular}{|l|l|}
\hline
\textbf{Status} & \textbf{Descrição} \\ \hline
PLACED & Indica que um pedido foi colocado no sistema. \\ \hline
CONFIRMED & Indica um pedido confirmado. \\ \hline
INTEGRATED & Indica um pedido que foi integrado no sistema. \\ \hline
CANCELLED & Indica um pedido que foi cancelado. \\ \hline
DISPATCHED & Indica um pedido que foi despachado ao cliente. \\ \hline
DELIVERED & Indica um pedido que foi entregue. \\ \hline
CONCLUDED & Indica um pedido que foi concluído. \\ \hline
\end{tabular}
    \fonte{\citeonline{iFood}}
\end{table}
 
  \begin{figure}[H]
    \centering
    \caption{Delivery Routes - Fluxo do status do pedido}
    \includegraphics[width=0.6\textwidth]{./dados/figuras/fig15}
    \fonte{\citeonline{iFood}}
    \label{fig:drStatusPedido}
\end{figure}

%%%Se o valor da coluna token_access é diferente ao hash da sessão siginfica que o usuário fez um novo login e gerou um novo hash. Neste caso podemos deslogar o usuário, caso os valores não se coinsidam.

Na \autoref{fig:drLogin} é possível visualizar o acesso ao módulo de gerenciamento, com autenticação apenas para administradores, realizada via e-mail e senha, cujo cadastro é efetuado por meio do link \textit{Register a new membership}. Habilitando a opção \textit{Remember Me}, o controle de única sessão por usuário é ativado, por intermédio de um \textit{hash} gerado no momento do login. É disponibilizado também o link \textit{I forgot my password} para realizar a recuperação da senha, mediante a confirmação de um e-mail existente na base de dados.

\begin{figure}[H]
    \centering
    \caption{Delivery Routes - Login}
    \includegraphics[width=0.4\textwidth]{./dados/figuras/fig7}
    \fonte{Autor}
    \label{fig:drLogin}
\end{figure}

%https://laravel.com/docs/5.7/hashing
O Laravel, juntamente com o Eloquent ORM, disponibiliza a implementação de autenticação de maneira muito simples, por meio do comando \textit{php artisan make:auth}, onde, com poucas instruções e linhas de código, é possível montar a estrutura de cadastro de usuário, recuperação de senha e memória de sessão de maneira extremamente segura, devido ao algoritmo \textit{Bcrypt}
\footnote{Método de criptografia do tipo \textit{hash} adaptativo para senhas que apresenta uma segurança maior em relação à maioria dos outros métodos criptográficos por meio da implementação da variável “custo”, que é proporcional à quantidade de processamento necessária para criptografar a senha \cite{bcrypt}.}
, utilizado para realizar a criptografia da senha.
%%% Conhecido como \textit{hash} adaptativo, pois pode permanecer resistente à ataques do tipo “força-bruta” com o tempo usando custos maiores de processamento

Ao realizar login, o usuário é encaminhado à \textit{dashboard} do sistema, uma interface gráfica que fornece visualização fácil e rápida dos principais indicadores de desempenho, atualizados em tempo real, obtendo assim valores e médias relevantes sobre o processo de negócios. 

Destaca-se na \autoref{fig:drDashboard}, além de indicadores de: Pedidos em aberto, Pedidos em rota, Motoboys disponíveis e Pedidos concluídos, uma barra de menus contendo: Cadastros (Motoboys e Pagamentos) e Movimentos (Entregas e Pedidos Em Aberto).

\begin{figure}[H]
    \centering
    \caption{Delivery Routes - Dashboard}
    \includegraphics[width=1.0\textwidth]{./dados/figuras/fig13}
    \fonte{Autor}
    \label{fig:drDashboard}
\end{figure}

\newpage
%http://repositorio.ufu.br/handle/123456789/22185
Este projeto foi desenvolvido em duas frentes: um desenvolvimento \textit{front-end} e outro \textit{back-end}. O \textit{front-end} foi baseado em um modelo, ou \textit{template}, chamado Pratt, uma implementação em Vue.js
\footnote{\textit{Framework} JavaScript progressivo, de código-aberto, para a criação de interfaces de usuário.} do tema AdminLTE 5.4 (desenvolvido com Bootstrap). Já o \textit{back-end} foi feito utilizando o \textit{framework} Laravel, na linguagem PHP.

O Bootstrap, é a biblioteca de componentes \textit{front-end} mais popular do mundo \cite{bootstrap}, e é voltado principalmente para o desenvolvimento de projetos mobile. Todos os componentes oferecidos, como alertas, cartões e formulários, são responsivos, ou seja, eles se redimensionam, se escondem, diminuem ou ficam maiores dependendo do tamanho do dispositivo em que a página é visualizada. Atualmente a biblioteca está na versão 4, o que trouxe muitas melhorias em relação a versão 3, como a simplificação dos componentes de navegação. No entanto, a versão usada pelo AdminLTE ainda é a 3.0 e por isso esta é a versão utilizada neste projeto.

O AdminLTE é um \textit{template} para desenvolvimento de \textit{dashboards} ou painéis de controle. Possui código-aberto, construído usando Bootstrap 3 e design responsivo. Por ser de código-aberto, dúvidas ou problemas encontrados podem ser postados no repositório do projeto no GitHub
\footnote{https://github.com/acacha/adminlte-laravel}, onde qualquer membro da comunidade pode prestar ajuda e propor soluções.

O Pratt é uma implementação em Vue.js baseado no \textit{template} responsivo AdminLTE, que por sua vez utiliza o \textit{framework} Bootstrap. Utilizar projetos como o Pratt como base para desenvolver outras aplicações reduz o tempo necessário para iniciar o desenvolvimento e ajuda a evitar possíveis problemas de configurações que um programador pode encontrar. 

Dessa forma, o Pratt vem com uma configuração pronta do Webpack (\autoref{fig:webpack}), um \textit{module blunder}, cujo principal objetivo, além de automatizar processos como a ofuscação, minificação e outras operações de pré-processamento de código, é permitir que o código seja escrito de forma modular, o que faz com que o desenvolvedor possa aproveitar todas as vantagens provindas do uso do \textit{module blunder}, sem ter que se preocupar com sua configuração inicial.

\begin{figure}[H]
    \centering
    \caption{Webpack - Module blunder}
    \includegraphics[width=0.8\textwidth]{./dados/figuras/fig17}
    \fonte{\citeonline{webpack}}
    \label{fig:webpack}
\end{figure}

\newpage
Ao selecionar a opção de menu “Motoboys”, o usuário consegue, então, gerenciá-los. Conforme é mostrado na \autoref{fig:drListaMotobys}, há uma tabela com o identificador, nome, CPF, e-mail, telefone e data de criação dos registros cadastrados na base de dados. A tabela também conta com botões para editar, excluir ou criar um registro. A edição do registro é apresentada na \autoref{fig:drEdicaoMotoby}.

\begin{figure}[H]
    \centering
    \caption{Delivery Routes - Lista de motoboys}
    \includegraphics[width=1.0\textwidth]{./dados/figuras/fig18}
    \fonte{Autor}
    \label{fig:drListaMotobys}
\end{figure}

\begin{figure}[H]
    \centering
    \caption{Delivery Routes - Edição do motoboy}
    \includegraphics[width=0.9\textwidth]{./dados/figuras/fig19}
    \fonte{Autor}
    \label{fig:drEdicaoMotoby}
\end{figure}

\newpage
Igualmente à tela de motoboys, a tela de pagamentos possui também uma tabela, dessa vez com o identificador e descrição dos registros cadastrados na base de dados, essa visualizada na \autoref{fig:drListaPagamentos}. A edição do registro é apresentada na \autoref{fig:drEdicaoPagamento}.

\begin{figure}[H]
    \centering
    \caption{Delivery Routes - Lista de pagamentos}
    \includegraphics[width=1.0\textwidth]{./dados/figuras/fig20}
    \fonte{Autor}
    \label{fig:drListaPagamentos}
\end{figure}

\begin{figure}[H]
    \centering
    \caption{Delivery Routes - Edição do pagamento}
    \includegraphics[width=0.9\textwidth]{./dados/figuras/fig21}
    \fonte{Autor}
    \label{fig:drEdicaoPagamento}
\end{figure}
No momento em que a tela retratada na \autoref{fig:drAPIorder} inicia, um serviço de consulta para capturar informações do pedido requerido. Esse serviço realiza uma requisição \textit{GET} na API 
\\ \textit{/order/9} do servidor, parametrizada para receber o código do pedido no estabelecimento, integrado pelo sistema anteriormente.

Nesse momento os possíveis \textit{status} do pedido poderem ser: \textit{INTEGRATED},
\\ \textit{CANCELLED}, \textit{DISPATCHED}, \textit{DELIVERED} ou \textit{CONCLUDED}.

\begin{figure}[H]
    \centering
    \caption{Delivery Routes - API - order}
    \includegraphics[width=1.0\textwidth]{./dados/figuras/fig22}
    \fonte{Autor}
    \label{fig:drAPIorder}
\end{figure}

Como resultado dessa consulta (como demonstra o \autoref{alg:getOrder}), o servidor entrega um objeto JSON com os respectivos dados solicitados pela \textit{route}, por meio da ferramenta  \textit{Resources}, exercendo a sua função como um camada de tratamento entre o Eloquent e as respostas JSON que são expostas pela API. Essas classes, criadas ao executar o comando \textit{php artisan make:resource OrdersResource} através do Artisan, permitem transformar facilmente, \textit{models} e \textit{collections} em JSON.

\begin{lstlisting}[caption={Delivery Routes - Route order}, style=htmlcssjs, label=alg:getOrder]
Route::get('/order/{id}', function ($order_id) {
    return new OrdersResource(Order::find($order_id));
});
\end{lstlisting}
% DESENVOLVIMENTO DA APLICAÇÃO-------------------------------------------------------------------

\chapter{DESENVOLVIMENTO DA APLICAÇÃO}
A proposta para este trabalho foi a construção de uma API que ficaria responsável
por aguardar e coletar as coordenadas geográficas dos pedidos gerados no estabelecimento,
juntamente com o desenvolvimento de uma aplicação encarregada de gerenciar os pedidos como entregas
com um painel de controle para geração de lotes de entregas e o desenvolvimento de um módulo destinado ao motoboy, no qual consegue observar, em um mapa, a rota e pontos de parada que devem ser percorridos no lote disponível para entrega, tudo isso em tempo real.

\section{Softwares utilizados}
Esta Seção apresenta os \textit{softwares} utilizados na etapa de codificação da aplicação.

\begin{itemize}
    \item PhpStorm: IDE completa de desenvolvimento para projetos codificados com a linguagem de programação PHP;
    \item Composer: gerenciador de pacotes em nível de aplicativo, que fornece um formato padrão para gerenciar dependências de software PHP e bibliotecas necessárias;
    \item Cmder: console para execução de linhas de comandos para o sistema operacional Windows, com essa ferramenta é possível executar comandos Unix (\textit{ls}, \textit{rm}, \textit{mv} e etc) diretamente no Windows;
    \item Laravel: \textit{framework} de programação baseada na linguagem PHP;
    \item Artisan CLI: interface de linha de comando incluída no Laravel, fornecendo vários comandos úteis durante a criação da aplicação, por exemplo: configuração de ambiente, verificar rotas, interagir com a aplicação e criar diversos tipos de arquivos (\textit{Migrations}, \textit{Controllers} e \textit{Models});
    \item Blade: ferramenta para criação de interface gráfica, utilizado pelo Laravel como uma ferramenta de \textit{template}, trazendo uma quantidade grande de funcionalidades que ajudam na criação de interfaces interativas;
    \item Eloquent ORM: ferramenta com funcionalidades que facilitam a inserção, atualização, busca e exclusão de registros diretamente no banco de dados;
    \item XAMPP: servidor independente de plataforma, que inclui: Apache, MySQL, phpMyAdmin, FileZilla FTP Server, OpenSSL, PHP e Perl;
    \item Jaspersoft Studio: ferramenta para projetar e executar modelos de relatório com  expressões, gráficos, mapas, tabelas e \textit{QR Codes}, criando documentos de qualquer complexidade a partir de informações presentes no banco de dados;
    \item GitHub: plataforma de hospedagem de código-fonte com controle de versão;
    \item Sourcetree: representação visual de repositórios da nuvem.
\end{itemize}

\newpage
Para desenvolvimento do servidor deste trabalho, foi escolhido trabalhar com Laravel, um \textit{framework} baseado na linguagem PHP de programação, em conjunto com o XAMPP, um servidor independente de plataforma para rodar sistemas localmente, que consiste principalmente na base de dados MySQL, o servidor web Apache e os interpretadores para linguagens de \textit{script}: PHP e Perl, o que facilita e agiliza o desenvolvimento. Como o conteúdo estará armazenado numa rede local, o acesso aos arquivos é realizado instantaneamente, ficando disponível no endereço \textit{http://127.0.0.1}.

Ao utilizar o comando \textit{composer create-project --prefer-dist laravel/laravel delivery-routes} no Cmder, dentro de uma pasta no sistema operacional destinada a programação da aplicação, foi criada a estrutura base do projeto, cedida pelo Laravel, apresentada na \autoref{fig:base-projeto}. A facilidade na criação do projeto se deve ao Cmder, que torna o trabalho no sistema operacional da Microsoft mais agradável, com essa ferramenta é possível rodar comandos do Linux e MAC que são baseados em Unix para o Windows. 

\begin{figure}[H]
    \centering
    \caption{Estrutura Laravel}
    \includegraphics[width=0.6\textwidth]{./dados/figuras/fig6}
    \fonte{Autor}
    \label{fig:base-projeto}
\end{figure}

\newpage
Foi necessária uma série de configurações de ambiente para que se tornasse possível o desenvolvimento da aplicação: primeiramente foi preciso instalar e configurar todas as dependências do PHP para que se pudesse ter acesso a linguagem de programação através do gerenciados de dependências Composer. Esta etapa contou com o auxílio do Cmder, terminal para Windows que reconhece comandos Unix e do Artisan, interface de linha de comando incluída no Laravel.

Foi optado, por questões de segurança, hospedar o código-fonte do projeto em nuvem, através da plataforma GitHub. A sincronia entre os arquivos locais e o GitHub foi possível com o auxílio do sistema de controle de versões chamado Git, representado visualmente pelo Sourcetree, que deixa o trabalho braçal por linha de comando (\textit{gitbash}) de lado, sendo também necessário sua instalação e configuração na máquina de desenvolvimento. 

O banco de dados escolhido para armazenar os registros foi o MySQL cuja instalação e configuração ocorreu automaticamente pelo XAMPP, que também é responsável pela instalação do servidor web local Apache. Por último, para ter acesso a manipulação do código, um editor de texto era necessário, optando-se pelo PHPStorm, uma IDE paga, mas estudantes podem conseguir licença estudantil, terminando assim a fase configuração de ambiente de desenvolvimento e garantindo aptidão para começar os trabalhos.

O próximo passo foi a modelagem dos dados, onde verificou-se que seria necessário a criação das tabelas: \textit{users}, \textit{motoboys}, \textit{deliveries}, \textit{payments} e \textit{orders}. Uma sexta tabela chamada de \textit{deliveries\_orders} precisou ser criada para vincular mais de um pedido à cada entrega, onde uma \textit{trigger}, disparada \textit{after update}, ficou responsável pela atualização do status de cada pedido vinculado à entrega, esta chamada de \textit{tr\_Status\_Orders}, apresentada na \autoref{fig:trigger}.

\begin{figure}[H]
    \centering
    \caption{Trigger tr\_Status\_Orders}
    \includegraphics[width=0.7\textwidth]{./dados/figuras/fig7}
    \fonte{Autor}
    \label{fig:trigger}
\end{figure}

\section{Coleta de dados}
O modelo solicitado para integração via requisição \textit{GET} na rota \textit{“/orders/opened”} na API do software principal do estabelecimento é exemplificado na \autoref{fig:drPlacedAPI}. O retorno da consulta deve ser em formato JSON, dispostos no formato “chave”: “valor” e contendo os seguinte atributos documentados pelo administrador: \textit{id}, \textit{latitude}, \textit{longitude}, \textit{payment\_id}, \textit{totalPrice}, \textit{prepaid} e \textit{status}.

\begin{figure}[H]
    \centering
    \caption{Delivery Routes - API - Coleta de dados}
    \includegraphics[width=0.6\textwidth]{./dados/figuras/fig14}
    \fonte{Autor}
    \label{fig:drPlacedAPI}
\end{figure}

\section{Módulo de gerenciamento}
 Inicialmente, o objetivo era construir uma aplicação completa para gestão de vendas, focando na geração do pedido e entrega, sendo essa para qualquer ramo de atividade. Porém descobriu-se a possibilidade de um nicho de mercado ainda não explorado e sem concorrentes, em que não é preciso disputar clientes com grandes empresas renomadas e maduras no assunto.
 
 A aplicação desenvolvida foca no gerenciamento das rotas de entregas dos pedidos gerados via e-PDV já operantes, necessitando apenas da disponibilização de uma API no formato de arquivo JSON (\autoref{fig:drHttpAPI}), essa compondo-se de dados essenciais para realização da integração automática abordada na próxima Seção.
 
  \begin{figure}[H]
    \centering
    \caption{Delivery Routes - Requisição HTTP}
    \includegraphics[width=0.6\textwidth]{./dados/figuras/fig16}
    \fonte{\citeonline{iFood}}
    \label{fig:drHttpAPI}
\end{figure}
 
 Primeiramente é preciso entender os possíveis status e o fluxo de um pedido integrado na Delivery Routes, ambos representados na \autoref{tab:drStatusPedido} e na \autoref{fig:drStatusPedido}.
 
 \begin{table}[H]
    \centering
    \caption{Delivery Routes - Status do pedido
    \label{tab:drStatusPedido}}
\begin{tabular}{|l|l|}
\hline
\textbf{Status} & \textbf{Descrição} \\ \hline
PLACED & Indica um pedido foi colocado no sistema. \\ \hline
CONFIRMED & Indica um pedido confirmado. \\ \hline
INTEGRATED & Indica um pedido que foi integrado no sistema. \\ \hline
CANCELLED & Indica um pedido que foi cancelado. \\ \hline
DISPATCHED & Indica um pedido que foi despachado ao cliente. \\ \hline
DELIVERED & Indica um pedido que foi entregue. \\ \hline
CONCLUDED & Indica um pedido que foi concluído. \\ \hline
\end{tabular}
    \fonte{\citeonline{iFood}}
\end{table}
 
 \begin{figure}[H]
    \centering
    \caption{Delivery Routes - Fluxo do status do pedido}
    \includegraphics[width=0.6\textwidth]{./dados/figuras/fig15}
    \fonte{\citeonline{iFood}}
    \label{fig:drStatusPedido}
\end{figure}

%%%Se o valor da coluna token_access é diferente ao hash da sessão siginfica que o usuário fez um novo login e gerou um novo hash. Neste caso podemos deslogar o usuário, caso os valores não se coinsidam.

Na \autoref{fig:drLogin} é possível visualizar o acesso ao módulo de gerenciamento, com autenticação apenas para administradores, realizada via e-mail e senha, cujo cadastro é efetuado por meio do link \textit{Register a new membership}. Habilitando a opção \textit{Remember Me}, o controle de única sessão por usuário é ativado, por intermédio de um \textit{hash} gerado no momento do login. É disponibilizado também o link \textit{I forgot my password} para realizar a recuperação da senha, mediante a confirmação de um e-mail existente na base de dados.

\begin{figure}[H]
    \centering
    \caption{Delivery Routes - Login}
    \includegraphics[width=0.4\textwidth]{./dados/figuras/fig7}
    \fonte{Autor}
    \label{fig:drLogin}
\end{figure}

%https://laravel.com/docs/5.7/hashing
O Laravel, juntamente com o Eloquent, disponibiliza a implementação de autenticação de maneira muito simples, por meio do comando \textit{php artisan make:auth}, onde, com poucas instruções e linhas de código, é possível montar a estrutura de cadastro de usuário, recuperação de senha e memória de sessão de maneira extremamente segura, devido ao algoritmo \textit{Bcrypt}
\footnote{Método de criptografia do tipo \textit{hash} adaptativo para senhas que apresenta uma segurança maior em relação à maioria dos outros métodos criptográficos por meio da implementação da variável “custo”, que é proporcional à quantidade de processamento necessária para criptografar a senha \cite{bcrypt}.}
, utilizado para realizar a criptografia da senha.

Ao realizar login, o administrador é apresentado à \textit{dashboard} do sistema, uma interface gráfica que fornece visualização fácil e rápida dos principais indicadores de desempenho, atualizados em tempo real, obtendo assim valores e médias relevantes sobre o processo de negócios. 

Destaca-se na \autoref{fig:drDashboard}, além de indicadores de: Pedidos Em Aberto, Pedidos Em Rota, Motoboys Disponíveis e Pedidos Concluídos, uma barra de menus contendo: Cadastros (Motoboys e Pagamentos) e Movimentos (Entregas e Pedidos Em Aberto).

\begin{figure}[H]
    \centering
    \caption{Delivery Routes - Dashboard}
    \includegraphics[width=1.0\textwidth]{./dados/figuras/fig13}
    \fonte{Autor}
    \label{fig:drDashboard}
\end{figure}

\newpage
Para desenvolver a \textit{dashboard}
%%% Rev-Madalozzo: listar todos os pacotes de terceiros AdminLTE e PDF


\section{Consulta de dados}
No momento em que a tela retratada na \autoref{fig:drAPIorder} inicia, um serviço de consulta para capturar informações do pedido requerido é executado. Esse serviço realiza uma requisição \textit{GET} na rota \textit{/order/\{id\}}, parametrizada para receber o código do pedido no estabelecimento, integrado pelo sistema anteriormente.

Nesse momento os possíveis status do pedido poderem ser: 
\\ \textit{INTEGRATED}, \textit{CANCELLED}, \textit{DISPATCHED}, \textit{DELIVERED} ou \textit{CONCLUDED}.

\begin{figure}[H]
    \centering
    \caption{Delivery Routes - API - order}
    \includegraphics[width=0.9\textwidth]{./dados/figuras/fig22}
    \fonte{Autor}
    \label{fig:drAPIorder}
\end{figure}

Como resultado dessa consulta (como demonstra o \autoref{alg:getOrder}), o servidor retorna o objeto requisitado (em formato JSON) com os respectivos dados solicitados pela rota, por meio da ferramenta \textit{Resources}, exercendo a sua função como um camada de tratamento entre o Eloquent ORM e as respostas JSON que são expostas pela API. Essas classes, criadas ao executar o comando \textit{php artisan make:resource OrdersResource} através do Artisan, permitem transformar facilmente, \textit{models} e \textit{collections} em JSON.

\begin{lstlisting}[caption={Delivery Routes - Route order}, style=htmlcssjs, label=alg:getOrder]
Route::get('/order/{id}', function ($order_id) {
    return new OrdersResource(Order::find($order_id));
});
\end{lstlisting}

\section{Módulo de entrega}
Neste módulo é apresentada à usabilidade do sistema por parte do motoboy, realizando a visualização da rota de entrega denominada para o momento.

\subsection{Tracking de entregas}
O primeiro status de uma entrega obrigatoriamente será \textit{CONFIRMED}, isso indica que a mesma foi planejada, visualizada e gerada pelo módulo de gerenciamento. Posteriormente a isso, a comanda (\autoref{fig:drCupom}) é impressa e então o fluxo de \textit{tracking} do pedido é iniciado (\autoref{fig:drFluxoPedido}).

 \begin{figure}[H]
    \centering
    \caption{Delivery Routes - Comanda do pedido}
    \includegraphics[width=0.4\textwidth]{./dados/figuras/fig24}
    \fonte{Autor}
    \label{fig:drCupom}
\end{figure}

\begin{figure}[H]
    \centering
    \caption{Delivery Routes - Fluxo do pedido}
    \includegraphics[width=0.85\textwidth]{./dados/figuras/fig25}
    \fonte{Adaptada de \citeonline{iFood}}
    \label{fig:drFluxoPedido}
\end{figure}

Neste cenário, ao ler o código QR impresso na comanda o motoboy é direcionado para a rota \textit{/deliveries/{id}/dispatched} (\autoref{alg:funcDispatched}), alterando seu status para \textit{DISPATCHED} e executando uma nova rota \textit{/deliveries/{id}/view}, onde o parâmetro solicitado para ambas é o código da entrega.

\begin{lstlisting}[caption={Delivery Routes - Função de despacho da entrega}, style=htmlcssjs, label=alg:funcDispatched]
public function dispatched($id) {
    $delivery = Delivery::find($id);
    $delivery->update(['status' => 'DISPATCHED']);
    return view('deliveries.view', compact('delivery'));
}
\end{lstlisting}

Logo após o despacho da entrega, é apresentada na tela no \textit{smartphone} do motoboy a representação, distância e tempo de deslocamento do percurso que deve ser percorrido na entrega desejada (\autoref{fig:drRotaEntregaInicio} e \autoref{fig:drRotaEntrega}).

\begin{figure}[H]
    \centering
    \caption{Delivery Routes - Despacho da entrega}
    \includegraphics[width=0.9\textwidth]{./dados/figuras/fig28}
    \fonte{Autor}
    \label{fig:drRotaEntregaInicio}
\end{figure}

\newpage
\begin{figure}[H]
    \centering
    \caption{Delivery Routes - Mapa da entrega}
    \includegraphics[width=0.8\textwidth]{./dados/figuras/fig27}
    \fonte{Autor}
    \label{fig:drRotaEntrega}
\end{figure}

\newpage
Para obter a rota entre dois pontos e exibi-la no mapa, é necessário utilizar o serviço da \textit{Google Directions} API. Lendo a sua documentação, verificou-se necessário a implementação das classes \textit{google.maps.DirectionsService} e o \textit{google.maps.DirectionsRenderer}.

Começando pelo \textit{DirectionsService}, o seu funcionamento é bem simples. Basta informar um objeto \textit{google.maps.DirectionsRequest}, o qual irá conter o ponto de origem e destino (\autoref{alg:currentPosition}), e o meio de transporte (carro, a pé, bicicleta ou transporte público), que ele irá retornar um objeto \textit{google.maps.DirectionsResult}, o qual contém as informações da rota, e o \textit{google.maps.DirectionsStatus}, que por sua vez define o estado final da requisição. Ele pode indicar sucesso (\textit{OK}), sem resultados (\textit{ZERO\_RESULTS}), erro (\textit{INVALID\_REQUEST} ou \textit{REQUEST\_DENIED}), etc.

\begin{lstlisting}[caption={Delivery Routes - Função de localização do usuário}, style=htmlcssjs, label=alg:currentPosition]
navigator.geolocation.getCurrentPosition(position);
function success(position) {
    currentPosition = new google.maps.LatLng(position.coords.latitude, position.coords.longitude);
};
\end{lstlisting}

Para adicionar os pontos de parada durante a rota, adquiridos através da rota \textit{/delivering/{id}} (\autoref{alg:pedidosEntrega}), que retorna todos os pedidos da entrega solicitada, é preciso informar uma lista de objetos \textit{google.maps.DirectionsWaypoint} (\autoref{alg:waypoints}), que são alguns pontos pré-definidos no meio do trajeto no objeto \textit{DirectionsRequest} antes de passá-lo para o \textit{directionsService.route}.

\begin{lstlisting}[caption={Delivery Routes - Route pedidos da entrega}, style=htmlcssjs, label=alg:pedidosEntrega]
Route::get('/delivering/{id}', function ($delivery_id) {
    return new OrdersResource(DB::table('orders')
    ->leftJoin('deliveries_orders', 'orders.id', '=', 
    'deliveries_orders.order_id')
    ->where('deliveries_orders.delivery_id', $delivery_id)->get());
});
\end{lstlisting}

\begin{lstlisting}[caption={Delivery Routes - Preenchimento dos pontos de parada}, style=htmlcssjs, label=alg:waypoints]
$.getJSON(json, function(pontos) {
    $.each(pontos.data, function(index, ponto) {
        waypoints[position] = {
            'location': new google.maps.LatLng(ponto.latitude, ponto.longitude)
        };
    });
});
\end{lstlisting}

\newpage
Já o \textit{DirectionsRenderer}, basicamente, fica responsável por renderizar o resultado fornecido pelo \textit{DirectionsService} (\autoref{alg:directionsService}).

\begin{lstlisting}[caption={Delivery Routes - Requisição de renderização do mapa}, style=htmlcssjs, label=alg:directionsService]
var request = {
    origin: currentPosition,
    waypoints: waypoints,
    destination: currentPosition,
    travelMode: google.maps.TravelMode.DRIVING
};

directionsService.route(request, function(result, status) {
    if (status == google.maps.DirectionsStatus.OK) {
        directionsDisplay.setDirections(result);
    }
});
\end{lstlisting}

O cálculo de distância e tempo (\autoref{alg:computeTotalDistance}), exibido na \autoref{fig:drRotaEntregaInicio}, é a última etapa. Nele é computado o tempo de deslocamento entre os pontos e também o tempo dos trâmites de entrega do produto para o cliente final, estimado em 5 minutos cada.

\begin{lstlisting}[caption={Delivery Routes - Função de cálculo do deslocamento}, style=htmlcssjs, label=alg:computeTotalDistance]
function computeTotalDistance(result) {
    var totalDist = 0;
    var totalTime = -300; // retorno
    var myroute = result.routes[0];
    for (i = 0; i < myroute.legs.length; i++) {
        totalDist += myroute.legs[i].distance.value;
        totalTime += myroute.legs[i].duration.value;
        totalTime += 300;
    }
}
\end{lstlisting}

A maioria dos produtos da Google são pagos, porém há um crédito (200 dólares para uso mensal gratuito, que condizem com até 28 mil carregamentos no valor de 7 dólares por milhar) disponível no momento da requisição dos serviços, que vai sendo debitado na medida que é requisitado. Não é cobrado valor algum até que esse crédito seja totalmente utilizado. Ele pode ser usado no \textit{Maps}, no \textit{Routes} ou no \textit{Places}. Para fins de uso e cobrança, é necessário uma chave da API JavaScript do \textit{Google Maps}. A chave da API é um identificador exclusivo usado para autenticar solicitações associadas ao projeto.

%\begin{algorithm}[H]
%O trecho de código fonte que é apresentado a seguir demonstra como é o serviço de                login do aplicativo: 
%, conforme apresenta a figura 19 
%, conforme a figura 20 mostra
%\caption{Delivery Routes - Página inicial}
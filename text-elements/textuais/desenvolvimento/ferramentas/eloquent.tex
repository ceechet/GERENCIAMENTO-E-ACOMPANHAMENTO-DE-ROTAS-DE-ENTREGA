%http://repositorio.ufu.br/handle/123456789/26374
%http://www.repository.ufrpe.br/handle/123456789/1427
O Eloquent é o \textit{Object Relational Mapper} (ORM) padrão do Laravel. É uma técnica de desenvolvimento de software usada para reproduzir tabelas de banco de dados sem formato de objetos relacionais, deixando a manutenção dos dados mais simples e fácil. Com esta técnica o programador não precisa escrever códigos na linguagem \textit{Structured Query Language} (SQL). Com a interface ORM é possível realizar operações de CRUD, além de criar as tabelas e montar relacionamentos 1:1 (um para um), 1:N (um para muitos) e N:M (muitos para muitos). Utilizando um padrão de registro ativo, o Eloquent transforma a tabela do banco de dados em uma estrutura MVC (\textit{Model-View-Controller}). Por exemplo, para listar todos as entregas da tabela de entregas ou encontrar uma entrega específica basta usar funções simples e intuitivas que equivalem à \textit{query} SQL desejada (\autoref{quadro:Eloquent}).

\begin{quadro}[H]
    \centering
    \caption{Comparativo SQL vs. Eloquent ORM
    \label{quadro:Eloquent}}
\begin{tabular}{|c|c|}
\hline
\textbf{SQL} & \textbf{Eloquent ORM} \\ \hline
select * from deliveries & Delivery::all() \\ \hline
select * from deliveries where id = 1 & Delivery::find(1) \\ \hline
\end{tabular}
    \fonte{Autor}
\end{quadro}

Segundo \citeonline{stauffer2017desenvolvendo}, o Eloquent ORM implementa o \textit{ActiveRecord}, o que significa que uma única classe (\textit{model}) é responsável por interagir com a tabela de forma completa, podendo inserir, atualizar e excluir novos registros.
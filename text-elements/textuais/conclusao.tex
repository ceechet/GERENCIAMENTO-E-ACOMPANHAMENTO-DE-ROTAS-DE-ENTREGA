\chapter{CONCLUSÃO E TRABALHOS FUTUROS}
O desenvolvimento do trabalho possibilitou explorar um grande acervo tecnológico, a partir da interação por meio de rotas com a \textit{Google Maps}, a qual automaticamente sugestões de percursos baseados no fluxo do trânsito do momento. Foi possível perceber, também que, por muitas vezes, é vantagem optar por essa tecnologia, pois ela abrange 99\% de cobertura no mundo e contempla uma documentação de fácil acesso para realizar a integração de aplicações web com as APIs de geolocalização. Por esse motivo, optou-se pelo desenvolvimento da aplicação em plataforma web, além do fato de existirem \textit{frameworks} como Laravel e Bootstrap que aceleram esse processo.

No transcorrer da elaboração, o tópico mais desafiador foi o cálculo de direções entre dois ou mais locais, por meio do \textit{Directions Service}, um serviço cedido pelo \textit{Google Maps JavaScript} API. Esse mecanismo recebe solicitações de direção e retorna um caminho eficiente. Nesse sentido, por utilizar \textit{waypoints} e \textit{current position}, sua etapa de desenvolvimento foi a mais complexa.

De acordo com a análise realizada durante o estudo dos conceitos utilizados, observou-se que as grandes empresas de comércio online pecam no pós-venda, uma vez que atrasam a entrega de seus produtos e instigam a insatisfação de seus clientes. Partindo dessa questão, foi possível criar um meio de melhorar essa logística e de recomendar aos usuários um planejamento.

Em meio ao cenário de enorme crescimento de pedidos online e \textit{delivery} de produtos, a aplicação Delivery Routes, desenvolvida no presente trabalho, é uma ferramenta com grande possibilidade de aceitação no mercado por diversos fatores, entre os quais destaca-se a oferta de um serviço integrado aos \textit{deliveries} já instalados e operantes nos estabelecimentos, de maneira organizada e documentada, o que possibilita maior praticidade para a \textit{software house} realizar a integração entre os serviços.

%O aplicativo permite também que o 
%A aplicação torna mais fácil para o 
%O aplicativo facilita também

Os trabalhos futuros terão como foco aprimorar as funcionalidades da aplicação, criando novas telas padrões (fale conosco, ajuda, sobre, configurações, entre outras) que compõem um aplicativo. 
Além disso, realizar a implementação de um algoritmo, a partir de um grafo, que mostre ao usuário como encontrar o menor caminho entre os pontos de origem e destino, incluindo os pontos de parada, é um dos principais processos planejados para o futuro desta implementação, permitindo ao usuário avaliar o possível percurso e tomar uma decisão. %Entretanto, o caminho planejado pode não ser o de menor comprimento e não ser o mais eficiente.

Para o módulo de entregas, por sua vez, será implementada uma nova \textit{feature}, que permitirá ao motoboy interagir com o método de navegação por voz durante o percurso, com suporte nos principais sistemas operacionais do mercado para não haver limitações. Serão trabalhadas melhorias de fluxos de processos, a fim de evitar possíveis problemas de desempenho. Do mesmo modo, será necessário alocar a aplicação em servidores externos para melhorar a capacidade de processamento com os cálculos de rotas.

Pretende-se, por fim, criar um modelo de autenticação baseado no padrão \textit{OAuth2} com \textit{client\_id} e \textit{client\_secret}, para identificação na API por meio de credenciais (\textit{token}). Nela, toda comunicação deverá ser realizada sob HTTPS. Um ambiente de homologação será oferecido para auxiliar no desenvolvimento.

Módulos futuros da aplicação contemplarão também a aproximação do cliente final com o estabelecimento. Neste módulo os clientes terão a facilidade de realizar pagamentos direto pelo aplicativo e informar o recebimento do produto. Com essa funcionalidade presente, adicionalmente será possível calcular precisamente o custo de cada deslocamento.
% INTRODUÇÃO-------------------------------------------------------------------

\chapter{INTRODUÇÃO}

Atualmente existem aplicações cruciais para facilitar a organização que um dia corrido necessita, como: lista de tarefas para você se lembrar de coisas que não podem ser esquecidas, GPS inteligente para guiar você sempre pelos melhores caminhos em sua cidade e previsão do tempo para nunca mais ser pego de surpresa na rua, são algumas funções de grande auxílio na vida de qualquer um. O princípio básico para desenvolvimento de uma aplicação é afirmar que um problema existe e precisa ser resolvido.

Locomoção é uma grande preocupação para quem vive em grandes centros urbanos. Utilizar o carro nem sempre é uma experiência agradável, afinal, não é nada difícil encontrar congestionamentos, principalmente quando é necessário atravessar a cidade para entregar ou buscar um objeto. É por esse motivo que a profissão chamada motoboy surgido provavelmente em São Paulo e disseminado em todo o país em 1998 \cite{MotoboyVeja}.

No cenário atual, é necessário uma otimização de rotas para estes profissionais da área de transportes que estão suscetíveis a perda de tempo caso não analisadas coerentemente por alguém capacitado. Um aplicativo móvel conectado a Google Maps Platform, proporcionando uma solução significativa na economia de tempo para os usuários.

Utilizando a Google Maps Platform, é possível encontrar o melhor trajeto até o destino com dados abrangentes e trânsito em tempo real. Oferecendo experiências personalizadas e ágeis que levam o mundo real até os usuários com mapas estáticos e dinâmicos. 

No segundo capítulo, são detalhadas as tecnologias utilizadas no desenvolvimento deste trabalho, como é o caso dos métodos de geolocalização disponíveis atualmente, o Sistema de Posicionamento Global Assistido (A-GPS) e o serviço de pesquisa e visualização de mapas e imagens de satélite da Terra gratuito na web Google Maps, fornecido pela Google.

O terceiro capítulo aborda o estudo das técnicas e ferramentas que foram necessárias para construção de toda a aplicação, passando pela linguagem base, servidor independente de plataforma, \textit{frameworks}, banco de dados e servidor web utilizados.

No quarto capítulo é descrito o desenvolvimento da aplicação Delivery Routes, assim como todos os passos realizados no decorrer do trabalho: captura dos dados, painel de gerenciamento para controle das entregas e o posterior acompanhamento das rotas de entregas.

Por fim, no quinto capítulo são apresentadas as conclusões e as propostas para futuros trabalhos.